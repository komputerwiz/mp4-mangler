%%%%%%%%%%%%%%%%%%%%%%%%%%%%%%%%%%%%%%%%%%%%%%%%%%%
%
%  New template code for TAMU Theses and Dissertations starting Spring 2021.
%
%
%  Author: Thesis Office
%
%  Last Updated: 1/13/2021
%
%%%%%%%%%%%%%%%%%%%%%%%%%%%%%%%%%%%%%%%%%%%%%%%%%%%
%%%%%%%%%%%%%%%%%%%%%%%%%%%%%%%%%%%%%%%%%%%%%%%%%%%%%%%%%%%%%%%%%%%%%
%%                           ABSTRACT
%%%%%%%%%%%%%%%%%%%%%%%%%%%%%%%%%%%%%%%%%%%%%%%%%%%%%%%%%%%%%%%%%%%%%

\chapter*{ABSTRACT}
\addcontentsline{toc}{chapter}{ABSTRACT} % Needs to be set to part, so the TOC doesn't add 'CHAPTER ' prefix in the TOC.

\indent Video media pervades our lives: movies and shows for entertainment, tutorials or documentaries for education, security footage, scientific monitoring, video calls for work or keeping in touch, recordings of cherished memories, etc. We capture this data in video files using various cameras and smartphones, but unfortunately, these file formats can be brittle and prone to corruption. Imagine opening an irreplaceable video only to see an error message, ``Sorry, this file is damaged and cannot be played.'' What causes this to happen? How can this be avoided? Can such a file be repaired? This research answers these questions for the MP4 file format, the most commonly used format in commercial and consumer devices at the time of writing. We identify the likely cause of unplayable files, present a preservation technique that adheres to the MP4 specification, and identify the necessary steps to restore the playability of these damaged files.

% \indent This is the first numbered page, lower case Roman numeral (ii). Page numbers are outside the prescribed margins, at the bottom of the page and centered; everything else is inside the margins.No bold on this page except the heading ABSTRACT if all major headings are bold. \emph{This \LaTeX ~ template applies to this exception}).
%
% Text begins two double spaces below the major heading. The Abstract should be no more than 350 words. Vertical spacing is double spaced. (\emph{This \LaTeX ~ template applies double space for this ABSTRACT.}) The margin settings and text alignment should be consistent throughout the document. There should be no numbered references or formal citations in ABSTRACT.
%
% The content of this ABSTRACT provides a complete, short snapshot of the research, addressing the purpose, methods, results, and conclusions of the document. As a result, it should stand alone without any formal citations or references to chapters/sections of the work. To accommodate with a variety of online database, images, complex equations, or Greek letters/symbols should also be avoided.This should be no longer than 350 words.
%
% The next pages are Dedication, Acknowledgments, Contributors and Funding Sources, and Nomenclature. Contributors and Funding Sources is required, the rest are optional.

\pagebreak{}